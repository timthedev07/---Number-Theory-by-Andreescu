\documentclass[11pt]{article}

\usepackage{sectsty}
\usepackage{graphicx}
\usepackage{amsmath}
\usepackage{amsthm}
\usepackage{float}
\usepackage{amssymb}

\newtheorem{theorem}{Theorem}
\newtheorem{example}{Example}

% Margins


\begin{document}

\section*{The Division Algorithm}

\begin{theorem}
  For any positive integers $a$ and $b$ there exists a unique pair of $(q, r)$ of nonnegative integers such that $$b = aq + r ,\ r < a$$
\end{theorem}

\begin{example}
  Prove that for all positive integers $n$, the fraction $$\frac{21n + 4}{14n + 3}$$ is irreducible.
\end{example}
\noindent Let $k = gcd(21n + 4, 14n + 3)$, then

\begin{align*}
  21n + 4 \equiv 0 \pmod{k} \\
  14n + 3 \equiv 0 \pmod{k} \\
\end{align*}
Multiplying the first equation by 2 an the second by 3 to match the coefficients of $n$

\begin{align*}
  42n + 8 \equiv 0 \pmod{k} \\
  42n + 9 \equiv 0 \pmod{k}
\end{align*}
$42n + 8$ and $42n + 9$ are consecutive integers and hence coprime, it then follows that $k = 1$.

\end{document}